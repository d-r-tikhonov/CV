%%%%%%%%%%%%%%%%%
% This is an sample CV template created using altacv.cls
% (v1.3, 10 May 2020) written by LianTze Lim (liantze@gmail.com). Now compiles with pdfLaTeX, XeLaTeX and LuaLaTeX.
% (v1.7.1b, 11 Jan 2024) forked by Nicolás Omar González Passerino (nicolas.passerino@gmail.com)
%
%% It may be distributed and/or modified under the
%% conditions of the LaTeX Project Public License, either version 1.3
%% of this license or (at your option) any later version.
%% The latest version of this license is in
%%    http://www.latex-project.org/lppl.txt
%% and version 1.3 or later is part of all distributions of LaTeX
%% version 2003/12/01 or later.
%%%%%%%%%%%%%%%%

%% If you need to pass whatever options to xcolor
\PassOptionsToPackage{dvipsnames}{xcolor}

%% If you are using \orcid or academicons
%% icons, make sure you have the academicons
%% option here, and compile with XeLaTeX
%% or LuaLaTeX.
% \documentclass[10pt,a4paper,academicons]{altacv}

%% Use the "normalphoto" option if you want a normal photo instead of cropped to a circle
% \documentclass[10pt,a4paper,normalphoto]{altacv}

%% Fork (before v1.6.5a): CV dark mode toggle enabler to use a inverted color palette.
%% Use the "darkmode" option if you want a color palette used to 
% \documentclass[10pt,a4paper,ragged2e,withhyper,darkmode]{altacv}

\documentclass[10pt,a4paper,ragged2e,withhyper]{altacv}

%% AltaCV uses the fontawesome5 and academicons fonts
%% and packages.
%% See http://texdoc.net/pkg/fontawesome5 and http://texdoc.net/pkg/academicons for full list of symbols. You MUST compile with XeLaTeX or LuaLaTeX if you want to use academicons.

%% Fork v1.6.5c: Overwriting sloppy environment to ignore any spaces and be used to solve overlapping cvtags
\newenvironment{sloppypar*}{\sloppy\ignorespaces}{\par}

% Change the page layout if you need to
\geometry{left=1.2cm,right=1.2cm,top=1cm,bottom=1cm,columnsep=0.75cm}

% The paracol package lets you typeset columns of text in parallel
\usepackage{paracol}

% Change the font if you want to, depending on whether
% you're using pdflatex or xelatex/lualatex
\ifxetexorluatex
  % If using xelatex or lualatex:
  \setmainfont{Roboto Slab}
  \setsansfont{Lato}
  \renewcommand{\familydefault}{\sfdefault}
\else
  % If using pdflatex:
  \usepackage[rm]{roboto}
  \usepackage[defaultsans]{lato}
  % \usepackage{sourcesanspro}
  \renewcommand{\familydefault}{\sfdefault}
\fi

% Fork (before v1.6.5a): Change the color codes to test your personal variant on any mode
\ifdarkmode%
  \definecolor{PrimaryColor}{HTML}{C69749}
  \definecolor{SecondaryColor}{HTML}{D49B54}
  \definecolor{ThirdColor}{HTML}{1877E8}
  \definecolor{BodyColor}{HTML}{ABABAB}
  \definecolor{EmphasisColor}{HTML}{ABABAB}
  \definecolor{BackgroundColor}{HTML}{191919}
\else%
  \definecolor{PrimaryColor}{HTML}{001F5A}
  \definecolor{SecondaryColor}{HTML}{0039AC}
  \definecolor{ThirdColor}{HTML}{F3890B}
  \definecolor{BodyColor}{HTML}{666666}
  \definecolor{EmphasisColor}{HTML}{2E2E2E}
  \definecolor{BackgroundColor}{HTML}{E2E2E2}
\fi%

\colorlet{name}{PrimaryColor}
\colorlet{tagline}{SecondaryColor}
\colorlet{heading}{PrimaryColor}
\colorlet{headingrule}{ThirdColor}
\colorlet{subheading}{SecondaryColor}
\colorlet{accent}{SecondaryColor}
\colorlet{emphasis}{EmphasisColor}
\colorlet{body}{BodyColor}
\pagecolor{BackgroundColor}

% Change some fonts, if necessary
\renewcommand{\namefont}{\Huge\rmfamily\bfseries}
\renewcommand{\personalinfofont}{\small\bfseries}
\renewcommand{\cvsectionfont}{\LARGE\rmfamily\bfseries}
\renewcommand{\cvsubsectionfont}{\large\bfseries}

% Change the bullets for itemize and rating marker
% for \cvskill if you want to
\renewcommand{\itemmarker}{{\small\textbullet}}
\renewcommand{\ratingmarker}{\faCircle}

%% sample.bib contains your publications
%% \addbibresource{main.bib}

\begin{document}
    \name{Dmitry Tikhonov}
    \tagline{3d year student}
    %% You can add multiple photos on the left or right
    \photoL{4cm}{drt}

    \personalinfo{
        \email{tikhonov.dr@phystech.edu}\smallskip
        \phone{+7 (918) 449-95-35}
        \location{Dolgoprudny, Russia}\\
        \github{d-r-tikhonov}
        %\homepage{nicolasomar.me}
        %\medium{nicolasomar}
        %% You MUST add the academicons option to \documentclass, then compile with LuaLaTeX or XeLaTeX, if you want to use \orcid or other academicons commands.
        % \orcid{0000-0000-0000-0000}
        %% You can add your own arbtrary detail with
        %% \printinfo{symbol}{detail}[optional hyperlink prefix]
        % \printinfo{\faPaw}{Hey ho!}[https://example.com/]
        %% Or you can declare your own field with
        %% \NewInfoFiled{fieldname}{symbol}[optional фhyperlink prefix] and use it:
        % \NewInfoField{gitlab}{\faGitlab}[https://gitlab.com/]
        % \gitlab{your_id}
    }
    
    \makecvheader
    %% Depending on your tastes, you may want to make fonts of itemize environments slightly smaller
    % \AtBeginEnvironment{itemize}{\small}
    
    %% Set the left/right column width ratio to 6:4.
    \columnratio{0.25}

    % Start a 2-column paracol. Both the left and right columns will automatically
    % break across pages if things get too long.
    \begin{paracol}{2}
        % ----- TECH STACK -----2
        % \cvsection{SOFT SKILLS}
        %     %% Fork v1.6.5c: The sloppypar* environment is used to avoid tags overlapping with section width
        %     \begin{sloppypar*}
        %         %% Fork 1.7.1b: Now in case you want to highlight any tag, just add a '/true' property next to its text and it will change tag's text and border colors.
        %         \cvtags{Quick learner/true, Responsible/true, Time managment/true, Teamwork/true, Flexibility, Stress-resistant, Attention to detail/true}
        %     \end{sloppypar*}
        % ----- TECH STACK -----
        
        % ----- LEARNING -----
        \cvsection{HARD SKILLS}
            \begin{sloppypar*}
                \cvtags{C, Python/true, git, GNS3/true, Linux, Bash/true, Markdown, TCP\slash IP/true, Cisco, VLAN/true, LACP, ARP/true, xSTP/true, RIP, OSPF}
            \end{sloppypar*}
        % ----- LEARNING -----
        
        % ----- LANGUAGES -----
        \cvsection{Languages}
            \cvlang{Russian}{Native}\\
            \divider

            \cvlang{English}{A2+}
            \bigskip
            %% Yeah I didn't spend too much time making all the
            %% spacing consistent... sorry. Use \smallskip, \medskip,
            %% \bigskip, \vpsace etc to make ajustments.
        % ----- LANGUAGES -----
            
        % ----- REFERENCES -----
        % \cvsection{References}
        %     \cvref{Prof.\ Alpha Beta}{Institute}{a.beta@university.edu}
        %     \divider

        %     \cvref{Boss\ Gamma Delta}{Business}{g.delta@business.com}
        % ----- REFERENCES -----
        
        % ----- MOST PROUD -----
        % \cvsection{Most Proud of}
        
        % \cvachievement{\faTrophy}{Fantastic Achievement}{and some details about it}\\
        % \divider
        % \cvachievement{\faHeartbeat}{Another achievement}{more details about it of course}\\
        % \divider
        % \cvachievement{\faHeartbeat}{Another achievement}{more details about it of course}
        % ----- MOST PROUD -----
        
        % use ONLY \newpage if you want to force a page break for
        % ONLY the current column
        \newpage
        
        %% Switch to the right column. This will now automatically move to the second
        %% page if the content is too long.
        \switchcolumn
        
        % ----- ABOUT ME -----
        \cvsection{About Me}
            \begin{quote}
                During my studies at MIPT I developed an interest networks. In my second year of study, I took an additional course on the basics of telecommunication technologies organized by the IITP RAS. This semester I was a student of the Klimanov M.M. computer networks course. I want to gain experience working with real networks that are located in different parts of the world.
            \end{quote}
        % ----- ABOUT ME -----
        
        % ----- EXPERIENCE -----
        % \cvsection{Experience}
        %     \cvevent{Charge}{Company}{Mm YYYY -- Mm YYYY}{City, Country}
        %     \begin{itemize}
        %         \item First achievement
        %         \item Second achievement
        %         \item Third achievement
        %     \end{itemize}
        %     \divider
            
        %     \cvevent{Charge}{Company}{Mm YYYY -- Mm YYYY}{City, Country}
        %     \begin{itemize}
        %         \item First achievement
        %         \item Second achievement
        %         \item Third achievement
        %     \end{itemize}
        % ----- EXPERIENCE -----
        
        % ----- EDUCATION -----
        \cvsection{Education}
            \cvevent{Moscow Institute Physics and Technology}{DREC}{2022 -- 2026}{Dolgoprudny, Russia}
            \begin{itemize}
                \item GPA: 8.7/10
            \end{itemize}
            \divider
            
            \cvevent{Fundamentals of telecommunication technologies}{The Institute for Information Transmission Problems}{2024}{MIPT}
            \begin{itemize}
                \item Lecturer: A. Kureev
            \end{itemize}

            \divider
            \cvevent{Advanced machine learning methods}{Digital Department}{2024}{MIPT}
            \begin{itemize}
                \item About course: solving analytical problems using python
            \end{itemize}
        % ----- EDUCATION -----
        
        % ----- PROJECTS -----
        \cvsection{Projects}
            \cvevent{Hamming-Code-Implementation}{\cvreference{\faGithub}{https://github.com/d-r-tikhonov/Hamming-Code-Implementation}}{March 2024 -- May 2024}{}
            \begin{itemize}
                \item A program for single-bit error correcting
            \end{itemize}
            \divider
        
        \cvevent{Random Number Generator}{\cvreference{\faGithub}{https://github.com/d-r-tikhonov/GRN_article}}{September 2023 -- April 2024}{}
        \begin{itemize}
            \item Random number generation based on bipolar transistor noise. The results of the work were presented in a scientific journal
        \end{itemize}
        \divider
        % ----- PROJECTS -----
    \end{paracol}
\end{document}